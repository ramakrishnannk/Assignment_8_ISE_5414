%% Based on a TeXnicCenter-Template by Tino Weinkauf.
%%%%%%%%%%%%%%%%%%%%%%%%%%%%%%%%%%%%%%%%%%%%%%%%%%%%%%%%%%%%%

%%%%%%%%%%%%%%%%%%%%%%%%%%%%%%%%%%%%%%%%%%%%%%%%%%%%%%%%%%%%%
%% HEADER
%%%%%%%%%%%%%%%%%%%%%%%%%%%%%%%%%%%%%%%%%%%%%%%%%%%%%%%%%%%%%
\documentclass[a4paper,oneside,12pt]{report}
% Alternative Options:
%	Paper Size: a4paper / a5paper / b5paper / letterpaper / legalpaper / executivepaper
% Duplex: oneside / twoside
% Base Font Size: 10pt / 11pt / 12pt


%% Language %%%%%%%%%%%%%%%%%%%%%%%%%%%%%%%%%%%%%%%%%%%%%%%%%
\usepackage[USenglish]{babel} %francais, polish, spanish, ...
\usepackage[T1]{fontenc}
\usepackage[ansinew]{inputenc}

\usepackage{lmodern} %Type1-font for non-english texts and characters


%% Packages for Graphics & Figures %%%%%%%%%%%%%%%%%%%%%%%%%%
\usepackage{graphicx} %%For loading graphic files
%\usepackage{subfig} %%Subfigures inside a figure
%\usepackage{pst-all} %%PSTricks - not useable with pdfLaTeX

%% Please note:
%% Images can be included using \includegraphics{Dateiname}
%% resp. using the dialog in the Insert menu.
%% 
%% The mode "LaTeX => PDF" allows the following formats:
%%   .jpg  .png  .pdf  .mps
%% 
%% The modes "LaTeX => DVI", "LaTeX => PS" und "LaTeX => PS => PDF"
%% allow the following formats:
%%   .eps  .ps  .bmp  .pict  .pntg


%% Math Packages %%%%%%%%%%%%%%%%%%%%%%%%%%%%%%%%%%%%%%%%%%%%
\usepackage{amsmath}
\usepackage{amsthm}
\usepackage{amsfonts}


%% Line Spacing %%%%%%%%%%%%%%%%%%%%%%%%%%%%%%%%%%%%%%%%%%%%%
%\usepackage{setspace}
%\singlespacing        %% 1-spacing (default)
%\onehalfspacing       %% 1,5-spacing
%\doublespacing        %% 2-spacing


%% Other Packages %%%%%%%%%%%%%%%%%%%%%%%%%%%%%%%%%%%%%%%%%%%
%\usepackage{a4wide} %%Smaller margins = more text per page.
%\usepackage{fancyhdr} %%Fancy headings
%\usepackage{longtable} %%For tables, that exceed one page
\usepackage{enumerate}

%%%%%%%%%%%%%%%%%%%%%%%%%%%%%%%%%%%%%%%%%%%%%%%%%%%%%%%%%%%%%
%% Remarks
%%%%%%%%%%%%%%%%%%%%%%%%%%%%%%%%%%%%%%%%%%%%%%%%%%%%%%%%%%%%%
%
% TODO:
% 1. Edit the used packages and their options (see above).
% 2. If you want, add a BibTeX-File to the project
%    (e.g., 'literature.bib').
% 3. Happy TeXing!
%
%%%%%%%%%%%%%%%%%%%%%%%%%%%%%%%%%%%%%%%%%%%%%%%%%%%%%%%%%%%%%

%%%%%%%%%%%%%%%%%%%%%%%%%%%%%%%%%%%%%%%%%%%%%%%%%%%%%%%%%%%%%
%% Options / Modifications
%%%%%%%%%%%%%%%%%%%%%%%%%%%%%%%%%%%%%%%%%%%%%%%%%%%%%%%%%%%%%

%\input{options} %You need a file 'options.tex' for this
%% ==> TeXnicCenter supplies some possible option files
%% ==> with its templates (File | New from Template...).



%%%%%%%%%%%%%%%%%%%%%%%%%%%%%%%%%%%%%%%%%%%%%%%%%%%%%%%%%%%%%
%% DOCUMENT
%%%%%%%%%%%%%%%%%%%%%%%%%%%%%%%%%%%%%%%%%%%%%%%%%%%%%%%%%%%%%
\begin{document}

\pagestyle{empty} %No headings for the first pages.


%% Title Page %%%%%%%%%%%%%%%%%%%%%%%%%%%%%%%%%%%%%%%%%%%%%%%
%% ==> Write your text here or include other files.

%% The simple version:
\title{Assignment VIII ISE 5414}
\author{Ramakrishnan Kalyanaraman\\
VT ID: 905782081\\
PID: rk126}
%\date{} %%If commented, the current date is used.
\maketitle

\begin{enumerate}
	\item[5.1.2] Let Min(t) be the Poisson process representing the minor defects over the length t. Let Maj(t) be the Poisson process representing the major defects over the length t.
	
	Now we want to calculate, $Pr[Min(t) + Maj(t) = k]$, where $X(t) = Min(t) + Maj(t)$ which represents number of defects, either major or minor, in the cable length of t.
	
	$Pr[Min(t) + Maj(t) = k]$ can be written as a convolution sum of $Min(t)$ and $Maj(t)$.
	
	$$Pr[Min(t) + Maj(t) = k] = \sum_{m = 0}^{t}{Pr[Min(t) = m] Pr[Maj(t) = k - m]}$$
	
	$$Pr[Min(t) + Maj(t) = k] = \sum_{m = 0}^{t}{\frac{(\alpha t)^m e^{-\alpha t}}{m!} \times \frac{(\beta t)^{k-m} e^{-\beta t}}{(k - m)!}}$$
	
	Multiplying and dividing by k! and taking all the "m" independent terms outside summation's scope.
	
	$$Pr[Min(t) + Maj(t) = k] = \frac{e^{-(\alpha + \beta)t}}{k!} \sum_{m = 0}^{t}{\frac{k!}{(k - m)! m!} (\alpha t)^m (\beta t)^{k - m}}$$
	
	This summation is actually a binomial theorem which leads to $(\alpha t + \beta t)^k$
	
	$$Pr[Min(t) + Maj(t) = k] = \frac{e^{-(\alpha + \beta)t}}{k!} (\alpha t + \beta t)^k$$
	
	$$Pr[X(t) = k] = \frac{e^{-(\alpha + \beta)t} [(\alpha + \beta)t]^k}{k!}$$
	
	Thus, proving that X(t) is a Poisson process of rate $(\alpha + \beta)$
	
	\item[5.1.7] Let k be the number of shocks that the system would have received till time t. Then,
	
	$$Pr[N(t) = k] = \frac{(\lambda t)^k e^{-\lambda t}}{k!}$$
	
	Also, we know that,
	
	Pr[System survives each shock] $= \alpha$
	
	Pr[System survives k shocks] $= \alpha^{k}$
	
	Pr[System is surviving at time t] $$= \sum_k{Pr[N(shocks) = k \cap N(t) = k]}$$
	
	Pr[System is surviving at time t] $$= \sum_k{Pr[N(shocks) = k|N(t) = k] Pr[N(t) = k]}$$
	
	Therefore, Pr[System is surviving at time t]
	
	$$ = \sum_k{\alpha^k \times \frac{(\lambda t)^k e^{-\lambda t}}{k!}}$$
	
	\item[5.1.9] We know that the Number of passengers arriving at the bus stop at time T, where T is a random variable  is a Poisson process and can be written as follows,
	
	$$Pr[X(T) = k|T = t] = \frac{(\lambda t)^k e^{-\lambda t}}{k!}$$
	
	$$Pr[X(T) = k|T = t] = \frac{(2 t)^k e^{-2 t}}{k!}$$
	
	Therefore, $$E[X(T)|T = t] = 2$$
	
	$$E[X(T)^2|T = t] = Var[X(T)|T = t] + E[X(T)|T = t]^2 = 2 + 2^2 = 6$$
	
	$$Pr[X(t) = k] = \int_{-\infty}^{\infty}{Pr[X(T) = k|T = t] f_T(t) dt}$$
	
	We know that,
	
	\[f_T(t) = \left\{ 
  \begin{array}{l l}
    1 & \quad \text{for $0 \leq t \leq 1$}\\
    0 & \quad \text{elsewhere}
  \end{array} \right.\]
	
	Therefore,
	
	$$Pr[X(t) = k] = \int_0^1{Pr[X(T) = k|T = t] dt}$$
	
	$$Pr[X(t) = k] = \int_0^1{\frac{(2 t)^k e^{-2 t}}{k!} dt}$$
	
	$$Pr[X(t) = k] = \frac{\Gamma(k + 1) - \Gamma(k + 1, 2)}{2k!} dt$$
	
	\item[5.1.12] We know that,
	
	$$Pr[X'(t) = j] = \int_0^{\infty}\frac{(\theta t)^k e^{-\theta t}}{k!} f(\theta) d\theta$$
	
	We are given,
	
	$f(\theta) = e^{-\theta}$ for $\theta > 0$
	
	$$Pr[X'(t) = j] = \int_0^{\infty}\frac{(\theta t)^k e^{-\theta t}}{k!} e^{-\theta} d\theta$$

	Solving this integral, we obtain,
	
	$$Pr[X'(t) = j] = \left(\frac{t}{1 + t}\right)^j \left(\frac{1}{1 + t}\right)$$
	
	For j = 0, 1, ...
	
	\item[5.2.1] We write the binomial distribution of $Pr[X(n, p) = 0]$ as,
	
	$$ Pr[X(n, p) = 0] = {n \choose 0} p^0 (1-p)^{n - 0} $$
	
	$$ Pr[X(n, p) = 0] = (1-p)^n $$
	
	We are given that, $np = \lambda$ i.e. $p = \lambda/n$
	
	$$ Pr[X(n, p) = 0] = \left(1-\frac{\lambda}{n}\right)^n $$

	Now, taking limit of n tending towards infinity on both sides,
	
	$$ \lim_{n\to\infty} Pr[X(n, p) = 0] = \lim_{n\to\infty} \left(1-\frac{\lambda}{n}\right)^n $$
	
	$$ \lim_{n\to\infty} Pr[X(n, p) = 0] = e^{-\lambda}$$
	
	Hence proved!
	
	In the second part,
	
	$$\frac{Pr[X(n, p) = k + 1]}{Pr[X(n, p) = k]} = \frac{{n \choose k + 1} p^{k + 1} (1 - p)^{n - k - 1}}{{n \choose k} p^{k} (1 - p)^{n - k}}$$
	
	On applying limit on both side with n tending towards infinity, we can get the desired output.
	
	$$\lim_{n\to\infty} \frac{Pr[X(n, p) = k + 1]}{Pr[X(n, p) = k]} = \frac{\lambda}{k + 1}$$
	
\end{enumerate}

\end{document}